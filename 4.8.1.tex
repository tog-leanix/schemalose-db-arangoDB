\subsection{Anwendungsszenarien}
Zur Betrachtung des praktischen Einsatzes von ArangoDB, haben wir drei verschiedene Anwendungsszenarien durchgeführt. Die Anwendungsszenarien sind so konzipiert, dass jedes Szenario einen Modelltyp des Multi Modells von ArangoDB darstellt. Des Weiteren untersuchen wir zwei verschiedene Zugriffsfunktionen - Java und FOXX. Als Ergebnis untersuchen wir die Zugriffszeit der jeweiligen Zugriffsfunktionen auf die Datenbank.
\subsubsection{Anwendungsszenario 1 - Dokumentenbasiertes Szenario}
Um das dokumentenbasierte Datenmodell zu testen, wurde als Anwendungsszenario die Suche nach einen gewissen Wertes in Dokumenten erstellt. Bei dem vorgestellten Schema wird in diesem Szenario die Farbe von gewissen Autos gesucht. Beispiel: Alle Autos der Farbe Rot.
\subsubsection{Anwendungsszenarien 2 - Key-Value Szenario}
In diesem Szenario geht es darum die Vorteile des Key-Value-Modells, daher wird mit dem Nummernschild ein gewisses Auto so schnell wie möglich gesucht.
\subsubsection{Anwendungsszenarien 3 - Graph Szenario}
Das dritte Szenario testet das Graph Modell von ArangoDB. Hierfür wollen wir in diesem Szenario erfahren, wie viele Unfälle ein Automodell (Bspw. VOLVO) an den anderen Automodellen angerichtet hat. Am Ende wollen wir eine Übersicht von Modellen und die Anzahl der Unfälle erhalten.