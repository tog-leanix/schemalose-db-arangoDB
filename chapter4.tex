\chapter{Kapitel IV - NoSQL – Ausgewählte Systeme in Action}
\section{ArangoDB}
\subsection{Anwendungsszenarien}
Zur Betrachtung des praktischen Einsatzes von ArangoDB, haben wir drei verschiedene Anwendungsszenarien durchgeführt. Die Anwendungsszenarien sind so konzipiert, dass jedes Szenario einen Modelltyp des Multimodells von ArangoDB darstellt. Des Weiteren untersuchen wir zwei verschiedene Zugriffsfunktionen - Java und FOXX. Als Ergebnis untersuchen wir die Zugriffszeit der jeweiligen Zugriffsfunktionen auf die Datenbank.
\subsubsection{Anwendungsszenario 1 - Dokumentenbasiertes Szenario}
Um das dokumentenbasierte Datenmodell zu testen, wurde als Anwendungsszenario die Suche nach einen gewissen Wertes in Dokumenten erstellt. Bei dem vorgestellten Schema wird in diesen Szenario die Farbe von gewissen Autos gesucht. Beispiel: Alle Autos der Farbe Rot.
\subsubsection{Anwendungsszenarien 2 - Key-Value Szenario}
In diesen Szenario geht es darum die Vorteile des Key-Value-Modells, daher wird mit dem Nummernschild ein gewisses Auto so schenll wie möglich gesucht.
\subsubsection{Anwendungsszenarien 3 - Graph Szenario}
Das dritte Szenario testet das Graph Modell von ArangoDB. Hierfür wollen wir in diesen Szenario erfahren, wie viele Unfälle ein Automodell (Bspw. VOLVO) an den anderen Automodellen angerichtet hat. Am Ende wollen wir eine Übersicht von Modellen und die Anzahl der Unfälle erhalten.
\subsection{Umsetzung in Java}
\subsubsection{Datenbankzugriff}
\subsubsection{Schnittstelle nach Außen}
\subsubsection{Anwendungsszenario 1}
\subsubsection{Anwendungsszenario 2}
\subsubsection{Anwendungsszenario 3}
\subsection{Umsetzung in FOXX}
\subsubsection{Datenbankzugriff}
FOXX ist ein Framework, welches von ArangoDB angebot wird. Es dient dazu Microservices direkt in der Datenbank zu implementieren. Daher sind die Datenzugriffe sehr Nahe an der Datenbank und besonders performant. Um Beispielsweise auf die Auto-Datenbankcollection zuzugreifen, schreibt man bei FOXX folgenden Code:
\lstinputlisting[linerange={5-5}]{./src/index.js}
Diese Collection wird in den folgenden Szenarien gebraucht, um Abfragen zu tätigen.
\subsubsection{Schnittstelle nach Außen}
Als Microservice bietet FOXX eine HTTP-Schnittstelle, um den Zugriff von anderen Systemen zu ermöglichen. Diese Schnittstelle wurde von dem ArangoDB-Team bereitgestellt und wird in JavaScript implementiert. Es können zwar NodeJS Pakete benutzt werden, aber es werden nicht alle unterstützt, da zum Beispiel der Zugriff aufs Dateisystem verweigert wird.
\subsubsection{Anwendungsszenario 1}
\lstinputlisting[linerange={15-30}]{./src/index.js}
\subsubsection{Anwendungsszenario 2}
\lstinputlisting[linerange={32-60}]{./src/index.js}
\subsubsection{Anwendungsszenario 3}
\lstinputlisting[linerange={62-87}]{./src/index.js}
\subsection{Anwendung und Test}
Screenshots Frontend 
\subsection{Fazit}