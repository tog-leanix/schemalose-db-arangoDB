\begin{abstract}
\section*{Zusammenfassung}\markboth{Zusammenfassung}{}
  \addcontentsline{toc}{chapter}{Zusammenfassung}
  Im Rahmen der Veranstaltung schemalose Datenbanken wird in dieser Arbeit auf allgemeine Programmierschnittstellen und die Datenbank ArangoDB eingegangen. 

Im ersten Teil zeigt die Arbeit weit verbreitete webbasierte Programmierparadigmen. Dabei geht es um Schnittstellenkommunikation über das Web. Außerdem wird hier erläutert, welche komplexe Datenstrukturen für diese Kommunikation verwendet werden und was die Vor- und Nachteile der Datenstrukturen, aber auch der Programmierparadigmen sind.

Im zweiten Teil bezieht sich die Arbeit auf die Datenbank ArangoDB. Hier wird gezeigt, aus welchen Komponenten die Datenbank besteht und in welchen typischen Anwendungsfällen ArangoDB eingesetzt wird. Außerdem zeigt die Arbeit welche Programmierschnittstellen und Anfragemechanismen die Datenbank anbieten.

Im letzten Teil wird dargestellt wie eine beispielhaft Installation und Inbetriebnahme von ArangoDB aussehen kann. Anhand mehrer ausgewählter Use-Cases wird erläutert, welche Vor- und Nachteile das Multi-Model der AranagoDB bietet.

\end{abstract}
